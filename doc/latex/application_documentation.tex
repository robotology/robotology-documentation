This is a template you can use to document an application (a bunch of scripts in a directory in icub-\/main/app, which instatiate and run a set of modules to produce a meaningful behavior of the robot).

This is a template you can use to document an application (a bunch of scripts in a directory in icub-\/main/app, which instantiate and run a set of modules to produce a meaningful behavior of the robot).

Replace \char`\"{}example\+Application\char`\"{} with the name of your application.

Look here to see the documentation produced by this code once it is parsed by Doxygen\+: example\+Application\textbackslash{}endref

\begin{DoxyVerb}/**
*
@ingroup icub_applications
\defgroup icub_exampleApplication exampleApplication

Place here a short description of the application. This will appear in the list
of the applications.

\section intro_sec Description
This application does not exist for real, it is just a template to be used
as a guideline for writing good documentation.

Place here a description of the application. You might want to use a list as in:

The application does:
-   this
-   that
-   ...

\section dep_sec Dependencies
List here a list of applications that are assumed to be up and running. For example
your application could assume iCubInterface and the attention system are running.

Example:

This module assumes \ref icub_exampleModule "exampleModule" is already running.

\section modules_sec Instantiated Modules
List here the modules that are instantiated by this application. This is useful to
browse the documentation of other modules. Example:
- \ref icub_exampleModule "exampleModule"

\section config_sec Configuration Files
Provide a comprehensive list of the configuration files. Usually located in ./conf. You
do not have to necessarely explain what each file does, as this should be already explained
in the documentation of each module. Link each file with the relative module it configures so
that it is possible to look up the documentation.

\section example_sec How to run the application
List here xml script(s) that allows running the application.

\author your_name

Copyright (C) 2008 RobotCub Consortium

CopyPolicy: Released under the terms of the GNU GPL v2.0.


This file can be edited at \in app/exampleApplication/doc.dox
**/\end{DoxyVerb}


This is a template you can use to document an application (a bunch of scripts in a directory in icub-\/main/app, which instatiate and run a set of modules to produce a meaningful behavior of the robot).

Replace \char`\"{}example\+Application\char`\"{} with the name of your application.

Look here to see the documentation produced by this code once it is parsed by Doxygen\+: example\+Application\textbackslash{}endref

\begin{DoxyVerb}/**
*
@ingroup icub_applications
\defgroup icub_exampleApplication exampleApplication

Place here a short description of the application. This will appear in the list 
of the applications.

\section intro_sec Description
This application does not exist for real, it is just a template to be used
as a guideline for writing good documentation.

Place here a description of the applciation. You might want to use a list as in:

The application does:
-   this
-   that
-   ...

\section dep_sec Dependencies
List here a list of applications that are assumed to be up and running. For example
your application could assume iCubInterface and the attention system are running.

Example:

This module assumes \ref icub_exampleModule "exampleModule" is already running.

\section modules_sec Instantiated Modules
List here the modules that are instantiated by this application. This is useful to 
browse the documentation of other modules. Example:
- \ref icub_exampleModule "exampleModule"

\section config_sec Configuration Files
Provide a comprehensive list of the configuration files. Usually located in ./conf. You 
do not have to necessarely explain what each file does, as this should be already explained
in the documentation of each module. Link each file with the relative module it configures so 
that it is possible to look up the documentation.
 
\section example_sec How to run the application
List here xml script(s) that allows running the application.
 
\author your_name

Copyright (C) 2008 RobotCub Consortium

CopyPolicy: Released under the terms of the GNU GPL v2.0.


This file can be edited at \in app/exampleApplication/doc.dox
**/\end{DoxyVerb}
 